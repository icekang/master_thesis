%%%%%%%%%%%%%%%%%%%%%%%%%%%%%%%%%%%%%%%%%%%%%%%%%%%%%%%%%%%
% EPFL report package, main thesis file
% Goal: provide formatting for theses and project reports
% Author: Mathias Payer <mathias.payer@epfl.ch>
%
% To avoid any implication, this template is released into the
% public domain / CC0, whatever is most convenient for the author
% using this template.
%
%%%%%%%%%%%%%%%%%%%%%%%%%%%%%%%%%%%%%%%%%%%%%%%%%%%%%%%%%%%
\documentclass[a4paper,11pt,oneside]{report}
% Options: MScThesis, BScThesis, MScProject, BScProject
\usepackage[MScThesis,lablogo]{EPFLreport}
\usepackage{xspace}

\title{A wonderful thesis\\about the merits of scientific writing}
\author{Naravich Chutisilp}
\supervisor{The Doctoral Student}
\adviser{Prof. Dr. sc. ETH Mathias Payer}
%\coadviser{Second Adviser}
\expert{The External Reviewer}

% \newcommand{\sysname}{FooSystem\xspace}

\dedication{
    \begin{raggedleft}
        It’s no use going back to yesterday, because I was a different person then.\\
        --- Lewis Carroll, Alice’s Adventures in Wonderland\\
    \end{raggedleft}
    \vspace{4cm}
    \begin{center}
        Dedicated to my lovely family.
    \end{center}
}
\acknowledgments{
% This is where you thank those who supported you on this journey. Good examples
% are your significant other, family, advisers, and other parties that inspired
% you during this project. Generally this section is about 1/2 page to a page.

% Consider acknowledging the use and location of this thesis package.

% Define your acknowledgments in \texttt{\textbackslash{}acknowledgments\{...\}}
% and show them with \texttt{\textbackslash{}makeacks}.

    I would like to show my deep gratitude to my mom and dad for their unwavering supports and firm trust to let me choose my own path, my two lovely sisters, Namking and Namkow, for their continuous love and understanding. 

    I am also grateful to all the friends I have made during my time at EPFL, Set, James, Kwang, Ice, Sundae, Nai, Cindy, Jenestin, Thomas, Paulina, Fah, Ting-Wei, Pin-Yen, Kai, Hong-Bin, Leo, Edvin, Anthon, Aamir, Jirka, and many more people I have not mentioned, who have made my life abroad memorable and amicable. Going to EPFL was the first time I have been abroad and I could not have asked for a better experience. Simple dinner in the evening every day after school, meaningful conversations about everything, trips to alpine mountains, and european cities.

    Furthermore, I could not forget the friends have made during my time at MIT, Mee, Pooh, and Cue who have made my final Master's semester fun and unforgettable. Coming to MIT was a dream come true, yet it was not easy to leave my friends in Switzerland and start anew in the US. However, since the first day I arrived, I have been welcomed with open arms and got shown to many places in Boston. Worries and fears have been replaced with excitement and joy. Hiking in the White Mountains, visiting California and silicon valley, walking along the freedom trail, enjoying delicacies of Lobster rolls and clam chowder.

    Additionally, it would not be possible if I did not have support from my friends in Thailand, V, Meak, Korn, Marie and uncountable more that I could not fit into this section. Calls and messages from you have always been a source of strength and comfort. Through though times and good times, you have always been there for me. Even though we are 9,188 km apart, or even 13,707 km apart, it feels like you are always by my side.

    I would like to thank Prof. Elazer Edelman and Karim Kaldry for giving me this unparalleled opportunity to join the IMES lab at MIT and let me complete my Master's thesis here, as well as providing me advises and support on my research. Opportunity 
}


\begin{document}
\maketitle
\makededication
\makeacks

\begin{abstract}
The sysname tool enables lateral decomposition of a multi-dimensional
flux compensator along the timing and space axes.

The abstract serves as an executive summary of your project.
Your abstract should cover at least the following topics, 1-2 sentences for
each: what area you are in, the problem you focus on, why existing work is
insufficient, what the high-level intuition of your work is, maybe a neat
design or implementation decision, and key results of your evaluation.
\end{abstract}

\begin{frenchabstract}
For a doctoral thesis, you have to provide a French translation of the
English abstract. For other projects this is optional.
\end{frenchabstract}

\maketoc

%%%%%%%%%%%%%%%%%%%%%%
\chapter{Introduction}
%%%%%%%%%%%%%%%%%%%%%%

The introduction is a longer writeup that gently eases the reader into your
% thesis~\cite{dinesh20oakland}. Use the first paragraph to discuss the setting.
In the second paragraph you can introduce the main challenge that you see.
The third paragraph lists why related work is insufficient.
The fourth and fifth paragraphs discuss your approach and why it is needed.
The sixth paragraph will introduce your thesis statement. Think how you can
distill the essence of your thesis into a single sentence.
The seventh paragraph will highlight some of your results
The eights paragraph discusses your core contribution.

This section is usually 3-5 pages.

%%%%%%%%%%%%%%%%%%%%
\chapter{Background}
%%%%%%%%%%%%%%%%%%%%

% The background section introduces the necessary background to understand your
% work. This is not necessarily related work but technologies and dependencies
% that must be resolved to understand your design and implementation.

% This section is usually 3-5 pages.

% Self-supervised learning
\section{Self-supervised learning}
Self-supervised learning (SSL) has been proven to be a promising approach for natural language processing tasks (NLP). SSL can be split into two parts which are pre-text tasks and downstream tasks. Pre-text tasks are tasks that are designed to learn representations from the data itself without human annotations. Downstream tasks are finetuning tasks that use the learned representations from pre-text tasks to solve the tasks that require human annotations. It allows models to learn representations while avoiding costs of human annotations~\cite{Jaiswal2020}. Unsupervised pre-text tasks is not uncommon in the computer vision (CV) domain. For example, autoencoder~\cite{Hinton2006}, denoising autoencoder~\cite{Vincent2008}, and colorization~\cite{Larsson2017} are some of the pre-text tasks that have been used in CV. However, suitable pre-text tasks for SSL in computer vision are still under research.

% Self-supervised learning in CVs
In the recent years, discriminative and generative tasks have been proposed as pre-text tasks for CV field. SimCLR~\cite{Chen2020Simple} is a discriminative method that learns representations by maximizing the similarity between differently augmented views of the same image and minimizing the similarity between views of different images. This is also known as contrastive learning (CL). Representations learned from SimCLR have shown to be effective for classification tasks, outperforming supervised learning on ImageNet~\cite{Russakovsky2015} using the same architecture ResNet50~\cite{He2016}. SimCLRv2~\cite{Chen2020} is an improvement of SimCLR using a larger encoder and projection head and momentum contrast (MoCo)~\cite{He2020} that uses a queue and a moving average encoder to stabilize the training. Nevertheless, contrastive learning methods require a large batch size and a large number of negative samples to work effectively~\cite{Chen2020Simple}. This makes it difficult to scale to large datasets and models. 

In contrast to contrastive learning, masked autoencoding (MAE) learns representations by generating pixel values of the masked out parts of images, honing similar spirit to masked language modeling (MLM)~\cite{Devlin2019} in NLP. It has been preliminarily explored on vision transformer (ViT) model at the same time the architecture has been proposed~\cite{Dosovitskiy2020vit}. However, the performace of masked autoencoding in the original paper is not as good as supervised learning. To make this works, He et al.~\cite{He2022} propose masking out 75\% of an image rather than 50\% experimented previously. Furthermore, an asymmetric encoder-decoder architecture of MAE is introduced. An encoder operates only on the visible pixels and a light weight decoder operates on the encoded tokens as well as the mask tokens. Compared to CL, batch size, data augmentation and number of negative samples are not as critical for MAE, rendering it more scalable. MAE outperforms supervised learning and supervised pre-training on ImageNet using the same architecture ViT. Additionally, MAE has been evaluated on segmentic segmentation on ADE20k~\cite{Zhou2018}, highlighting that it can be used for dense prediction downstream tasks as well. Nonetheless, working on the pixel level, MAE is computationally expensive compared to CL.

Along MAE, bidirectional encoder for image transformer (BEiT) first learns image patch tokenizer and predicts the tokenized values of the mask tokens instead of the pixel values in MAE~\cite{Bao2022beit}. BEiT has been shown to outperform supervised learning and supervised pre-training on ImageNet. It has also been evaluated on segmentic segmentation on ADE20k achieving state-of-the-art performace at the time of the paper submission. As BEiT requires tokenizer to be learned, it adds more complexity to the pre-text task compared to MAE. DINOv2\cite{Oquab2024dinov} proposes to learn representations by predicting the tokenized values of the mask tokens as well as matching the class tokens of teacher and student networks on different crops of the same image. This approach has been shown to outperform other image SSLs at the time. Performing representation learning at the token level allows DINOv2 to be more efficient than pixel-level SSLs. In addition, I-JEPA~\cite{Assran2023} adapts this idea one step further. Attempting to find the most efficient approach to pre-train the, I-JEPA proposes to predict directly the encoded vectors of mask tokens instead of the tokenized values. Evaluated on image classification, I-JEPA has shown to be comparable to other SSLs while being simpler and more efficient. 

% Self-supervised learning in medical imaging
\section{Self-supervised learning in medical imaging}
In medical imaging, self-supervised learning would be highly beneficial as human annotations for medical images are expensive and time-consuming. However, as it is a specialized domain, recent SSL methods in CV are not yet fully explored. There are challenges in applying SSL in medical imaging. First, medical images can be in 3D, which is not common explored in SSL for natural which is mostly in 2D. Second, medical images can be in various modalities such as X-ray, CT, MRI, and ultrasound. Specifically, in our study, we focus on optical coherence tomography (OCT) images of arteries which are 3D and does not have any public datasets for SSL. This poses a challenge for us as there is no direct research directly addressing our problem.

% Basic rotation, jigsaw, rubik
 Following the same idea, Song et al.\cite{Song2022} use rotation prediction, instance discrimination and VAE~\cite{Kingma2013}. Using VAE as a pre-text task, the model is semgnetaion-ready for COVID infection segmentation on lung CT images. This method outperforms supervised learning on similar architecture U-Net\cite{Ronneberger2015} and U-Net++\cite{Zhou2020}. Approaching 3D medical images differently, Rubik's cube solving is used as a 3D pre-text task where the model learns to classifies the orientation and ordering of a shuffled volumetric cube. Even though discriminative, Rubik's cube shows to improvement 3D segmentation in medical images \cite{Zhuang2019}. Jigsaw learns feature representations by predicting the correct order of shuffled patches\cite{Noroozi2016}. Deep clustering clusters features vectors of the image in an unsupervised manner and uses thes clusters as pseudo-labels for classification\cite{Caron2018}. Applying one or multiple pre-text tasks has been the main theme in several medical imaging SSL research papers \cite{Zhou2021, Zhang2021, Dufumier2021}. Taleb et al. \cite{Taleb2020} study 5 separated 3D self-supervised learning tasks, namely predicting latent vectors of adjacent patches, predicting the location of a given patch, solving jigsaw, predicting rotation angle and contrastive learning. In their study, predicting latent vectors of adjacent patches yields the best results in downstream tasks.

% Constrastive learning
TS-SSL~\cite{Zhang2021} proposes an SSL on 2D spectral domain optical coherence tomography (SD-OCT) images of retina to better classify retinal anomaly. It learns representation from classification labels, discriminative and generative tasks simultaneously. Using contrastive loss, it maximizes the similarity between rotated views or shuffled patches of the same image and minimizes the similarity between views of different images. At the same time, it uses a same representation to predict the rotation angle of the rotated views and the order of the shuffled patches. Along these tasks, TS-SSL also uses classification labels to predict the class of the image. However, this method only works better than supervised learning when 10\% of the labels are used for training. TS-SSL is not the only method that adopts modern SSL methods to medical imaging. Dufumier et al.\cite{Dufumier2021} adds anoter layer to contrastive learning by using meta-data available in medical images. Specifically, they use age of the patient to indicate the degree of similarity between different images. This method improves bipolar disorder, Alzheimer and schizophrenia classification on 3D brain MRI images. Similarly, CLIP loss can be used to leverage multi-modalities for representation learning \cite{Hager2023}. Encouraging images and their tabular meta-data, ubiquitous in medical imaging, to have similar representations. This method has been shown to improve classification tasks on brain MR images. 

Exploring boon of contrastive learning in segmentation, He et al.~\cite{He2022Intra} propose an intra- and inter-slice contrastive learning for point supervised OCT fluid segmentation of retina. The nature of this retinal OCT is 3D. While training segmentation model with U-Net~\cite{Ronneberger2015}, the authors leverage the fact that adjacent slices of the same volume are highly correlated, inter-slice CL is proposed to maximize the similarity between encoded vectors of adjacent slices as well as their segmentation masks. Segmentation masks are also compared with their ground truth masks and cross-entropy loss. Segmented masks are subsequently used to select the locations of small patches to be used for intra-slice CL. Intra-slice CL maximizes the similarity between encoded fluid patches and the backgroun fluid patches. Its result outperforms other point-based segmentation methods but is not as good as fully supervised learning. Besides spatio consistency as previous work, temporal consistency can also be imposed\cite{Ren2022}. Given that images are taken from the same patient at different time points, temporal consistency can be used to improve the segmentation of the images. This method has been shown to improve the segmentation of brain MRI images.

% Restorative tasks
Alternatively, generative tasks have been proposed as pre-text tasks for medical imaging. Patch shuffling\cite{Chen2019} propose a restorative task for 2D medical images of MRI, CT and ultrasound. First patches are randomly cut from the original image and placed to random locations. The model learns to restore the images back to their original versions. This method improves the downstream segmentation tasks in a data-scarce setting. In-painting\cite{Pathak2016} removes a part of the image and the model learns to predict the missing part. Advancing further, Genesis\cite{Zhou2021} explore image distortion algorithms for 3D restorative pre-text tasks on CT images. Thorough experiments show that Genesis outperforms specialized 3D state-of-the-art segmentation models and other pre-text tasks including de-noising\cite{Vincent2010}, in-painting\cite{Pathak2016}, jigsaw\cite{Noroozi2016}, deep clustering\cite{Caron2018}, rubik's\cite{Zhuang2019} and patch shuffling\cite{Chen2019}. This establishes a new base line for 3D SSL in medical imaging showing that mixing pre-text tasks is beneficial.

Introducing deep clustering\cite{Caron2018} idea to Genesis\cite{Zhou2021}, TransVW adds a classification task to the restorative task of Genesis. First, visual words are mined automatically and grouped into clusters with deep latent features. These visual words (VW) are used to be classified and restored. Visual words improve segmentation and classification task in some datasets compared to Genesis\cite{Haghighi2021}. DiRA\cite{Haghighi2024} further improves TransVW by adding adversarial model to tell apart the restored images and the original image. Additionally, constrastive learning is used as an additional pre-text task. Combining, discriminative, restorative and adversarial training, DiRA outperforms TransVW and train-from-scratch models. Eventhough, TransVW and DiRA have shown to be better than Genesis, their evaluations are not as thorough. TransVW only outperforms Genesis in some datasets, while DiRA only evaluates its performace against TransVW.

% 2D and 3D self-supervised learning (extension to 3D)
\section{Self-supervised learning alternatives}
Self-supervised learning has a great potential in medical imaging. If the simple and robust pre-text tasks can be found, it can be used to learn representations from medical images without human annotations. However, such tasks are being explored and much work on the evaulation have to be done. Alternatives are introduced to efficiently train the models with existing small medical datasets.

SuPReM suggests to pre-train models with supervised learning on a large annotated datasets. SuPReM collates 9,262 CT volumes, pre-trains a models to segment subset of organs and fine-tune them on mutually exclusive subset of organs. The experiments show that it requires less data to learn meaning representations than SSL while also provides better transferability\cite{Li2024}.
% 3D self-supervised learning in medical imaging
% Genesis, SwinUNTER

% supervised segmentation in medical imaging
% nnUNet, Segformer?

%%%%%%%%%%%%%%%%
\chapter{Design}
%%%%%%%%%%%%%%%%

Introduce and discuss the design decisions that you made during this project.
Highlight why individual decisions are important and/or necessary. Discuss
how the design fits together.

This section is usually 5-10 pages.


%%%%%%%%%%%%%%%%%%%%%%%%
\chapter{Implementation}
%%%%%%%%%%%%%%%%%%%%%%%%

The implementation covers some of the implementation details of your project.
This is not intended to be a low level description of every line of code that
you wrote but covers the implementation aspects of the projects.

This section is usually 3-5 pages.


%%%%%%%%%%%%%%%%%%%%
\chapter{Evaluation}
%%%%%%%%%%%%%%%%%%%%

In the evaluation you convince the reader that your design works as intended.
Describe the evaluation setup, the designed experiments, and how the
experiments showcase the individual points you want to prove.

This section is usually 5-10 pages.


%%%%%%%%%%%%%%%%%%%%%%
\chapter{Related Work}
%%%%%%%%%%%%%%%%%%%%%%

The related work section covers closely related work. Here you can highlight
the related work, how it solved the problem, and why it solved a different
problem. Do not play down the importance of related work, all of these
systems have been published and evaluated! Say what is different and how
you overcome some of the weaknesses of related work by discussing the 
trade-offs. Stay positive!

This section is usually 3-5 pages.


%%%%%%%%%%%%%%%%%%%%
\chapter{Conclusion}
%%%%%%%%%%%%%%%%%%%%

In the conclusion you repeat the main result and finalize the discussion of
your project. Mention the core results and why as well as how your system
advances the status quo.

\cleardoublepage
\phantomsection
\addcontentsline{toc}{chapter}{Bibliography}
\printbibliography

% Appendices are optional
% \appendix
% %%%%%%%%%%%%%%%%%%%%%%%%%%%%%%%%%%%%%%
% \chapter{How to make a transmogrifier}
% %%%%%%%%%%%%%%%%%%%%%%%%%%%%%%%%%%%%%%
%
% In case you ever need an (optional) appendix.
%
% You need the following items:
% \begin{itemize}
% \item A box
% \item Crayons
% \item A self-aware 5-year old
% \end{itemize}

\end{document}